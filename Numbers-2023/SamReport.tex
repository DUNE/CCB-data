\section{Storage Characteristics}

\begin{table}[ht]
 \centering\csvautotabularright{external/mc.csv}
 \label{tab:MCinSAM}
\caption{Summary  of total simulation in sam by detector type as of Nov 1, 2022.}
\end{table}

\begin{table}[ht]
 \centering\csvautotabularright{external/detector.csv}
 \label{tab:DataInSam}
\caption{Summary  of total detector data in sam by detector type as of Nov 1, 2022.}
\end{table}
\clearpage

\begin{table}[ht]
 \centering\csvautotabularright{external/TOPTYPES.csv}
 \label{tab:ProtoDUNEsizes}
\caption{Classification of the largest data samples in sam.  They are classified as detector(data) or mc, by the detector producing the data, by the stream (readout time) and by the data tier.  Some types, test and noise for example are archival only.  }
\end{table}

\begin{table}[ht]
\centering\csvautotabularright{external/DUNERSEUSAGE-2022-11-14.csv}
 \label{tab:RSEUsage}
\caption{Summary  of DUNE disk areas known to rucio \cite{scotgrid}.  The CASTOR and FNAL dCache areas are partially tape-backed and expandable. }
\end{table}
\clearpage