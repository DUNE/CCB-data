\documentclass[12pt,landscape]{article}
\usepackage{csvsimple}
\usepackage{graphicx}
\usepackage{hyperref}
\parindent=0pt
\setlength{\textwidth=9.5in}
\setlength{\oddsidemargin=-.25 in}
\setlength{\topmargin=0 in}
\setlength{\textheight=6 in}
\title{DUNE Offline Computing Model Calculations}
\author{H. Schellman for the Computing Consortium}
\date{\today}
\begin{document}


\makeatletter
\csvset{
  autotabularright/.style={
    file=#1,
    after head=\csv@pretable\begin{tabular}{|*{\csv@columncount}{r|}}\csv@tablehead,
    table head=\hline\csvlinetotablerow\\\hline,
    late after line=\\,
    table foot=\\\hline,
    late after last line=\csv@tablefoot\end{tabular}\csv@posttable,
    command=\csvlinetotablerow},
}
\makeatother
\newcommand{\csvautotabularright}[2][]{\csvloop{autotabularright={#2},#1}}

\maketitle
\section{Introduction}

This is an annual projection for DUNE CPU and storage needs intended for use at the Computing Contribitions Board meeting in November 2022. It projects needs for 2023 onwards. 

The projection is done using codes at: \href{https://github.com/DUNE/CCB-data/tree/master/Numbers-2023}{https://github.com/DUNE/CCB-data/tree/master/Numbers-2023} from parameters stored in a json file. We use CPU and storage sizes derived from protoDUNE and simulation experience and apply them to projected numbers of events from the various DUNE detectors. 

Changes since the last report include:

\begin{itemize}
\item a later start for ProtoDUNE 2 running at CERN
\item use of slot time instead of CPU time as our codes often require more memory than is available for a single batch slot. 
\item revisions to near term use based on the 2022 experience
\end{itemize}

\section{Storage Characteristics}

\begin{table}[ht]
 \centering\csvautotabularright{external/mc.csv}
 \label{tab:MCinSAM}
\caption{Summary  of total simulation in sam by detector type as of Nov 1, 2022.}
\end{table}

\begin{table}[ht]
 \centering\csvautotabularright{external/detector.csv}
 \label{tab:DataInSam}
\caption{Summary  of total detector data in sam by detector type as of Nov 1, 2022.}
\end{table}
\clearpage

\begin{table}[ht]
 \centering\csvautotabularright{external/ProtoDUNE_sizes.csv}
 \label{tab:ProtoDUNEsizes}
\caption{Summary  of ProtoDUNE data samples in sam.}
\end{table}

\begin{table}[ht]
\centering\csvautotabularright{external/DUNERSEUSAGE-2022-11-14.csv}
 \label{tab:RSEUsage}
\caption{Summary  of DUNE disk areas known to rucio \cite{scotgrid}.  The CASTOR and FNAL dCache areas are partially tape-backed. }
\end{table}
\clearpage

\section{Model Assumptions}
