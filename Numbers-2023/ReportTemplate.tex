\documentclass[12pt]{article}
\usepackage{csvsimple}
\begin{document}

\makeatletter
\csvset{
  autotabularright/.style={
    file=#1,
    after head=\csv@pretable\begin{tabular}{|*{\csv@columncount}{r|}}\csv@tablehead,
    table head=\hline\csvlinetotablerow\\\hline,
    late after line=\\,
    table foot=\\\hline,
    late after last line=\csv@tablefoot\end{tabular}\csv@posttable,
    command=\csvlinetotablerow},
}
\makeatother
\newcommand{\csvautotabularright}[2][]{\csvloop{autotabularright={#2},#1}}


 \csvautotabularright{CoresbyDet.csv}
 
 \csvautotabularright{Disk_by_location.csv}
  \csvautotabularright{Tape_by_location.csv}
%\begin{tabular}{r r r r r r r r r r r r r r r r r r r r r r r }
%\csvreader[head to column names]{CoresbyDet.csv}{}
%\end{tabular}
 \csvautotabularright{Table-Cores.csv}
 \csvautotabularright{Table-Events.csv}
 \csvautotabularright{Table-HS06.csv}
 \csvautotabularright{Table-Raw.csv}
 \csvautotabularright{Table-Reco-CPU.csv}
 \csvautotabularright{Table-Reco.csv}
 \csvautotabularright{Table-Sim-Events.csv}
 \csvautotabularright{Table-Sim-CPU.csv}
 \csvautotabularright{Table-Sim.csv}
 \csvautotabularright{Table-Test.csv}
 \csvautotabularright{Table-Total-CPU.csv}
 \csvautotabularright{Table-WALL.csv}
\end{document}