\documentclass[12pt]{article}
\renewcommand{\familydefault}{\sfdefault}
\usepackage{csvsimple}
\usepackage{graphicx}
\usepackage{hyperref}
\parindent=0pt
\parskip=5 pt
\setlength{\textwidth=7in}
\setlength{\oddsidemargin=-.25 in}
\setlength{\topmargin=0 in}
\setlength{\textheight=8.5 in}
\usepackage{draftwatermark}
\SetWatermarkText{Draft}
\SetWatermarkScale{5}
\title{DUNE Offline Computing Model Calculations}
\author{H. Schellman for the Computing Consortium}
\date{\today}
\begin{document}


\makeatletter
\csvset{
  autotabularright/.style={
    file=#1,
    after head=\csv@pretable\begin{tabular}{|*{\csv@columncount}{r|}}\csv@tablehead,
    table head=\hline\csvlinetotablerow\\\hline,
    late after line=\\,
    table foot=\\\hline,
    late after last line=\csv@tablefoot\end{tabular}\csv@posttable,
    command=\csvlinetotablerow},
}
\makeatother
\newcommand{\csvautotabularright}[2][]{\csvloop{autotabularright={#2},#1}}

\maketitle
\section{Introduction}

This is an annual projection for DUNE CPU and storage needs intended for use at the Computing Contributions Board meeting in December 2022. It projects needs for 2023 onwards. 

The overall computing model  and 2022 projections for DUNE are described in chapters 6-13 of the recent (Oct. 2022) DUNE Conceptual Design Report \cite{DUNE:2022fcw}.   This 

The projection is done using codes at: \href{https://github.com/DUNE/CCB-data/tree/master/Numbers-2023}{https://github.com/DUNE/CCB-data/tree/master/Numbers-2023} from parameters stored in a json file. We use CPU and storage sizes derived from protoDUNE and simulation experience and apply them to projected numbers of events from the various DUNE detectors. 

Details are provided in the appendices while the main body of this note summarizes pledges, usage and projected need for the CCB.

Changes since the last report include:

\begin{itemize}
\item A later start for ProtoDUNE 2 running at CERN.
\item Use of slot time instead of CPU time as our codes often require more memory than is available for a single batch slot. 
\item Revisions to near-term requests based on the 2022 experience including a hold on tape requests from the collaboration during protoDUNE activities. 
\end{itemize}


\section{Disk and Tape}

Generally, raw data are stored on tape at both CERN and FNAL.  Simulation and reconstructed data  have one tape copy at Fermilab and recent reconstructed and simulated samples have one (or two) disk copies with one at Fermilab and one in Europe.  Appendix \ref{storage} gives details on the size and types of data from the SAM data catalog.

\subsection{Disk}
Table \ref{tab:RSEUsage} summarizes the disk utilization known to rucio.  Some sites, notably IN2P3 and TIFR, are not yet fully integrated so do not show up in the rucio reports, while others (PIC) are still being filled.

Figure/Table  \ref{fig:Cumulative-Disk}  summarize the cumulative disk projected by our model. These numbers are used to generate the request for 2023. 

Table \ref{tab:DiskPledges} summarizes the pledges from previous years compared to the actual amounts allocated and used from Table \ref{tab:RSEUsage} .   The 2023 request has been re-evaluated in light of underuse in 2021 and 2022 and should better match the likely capacity of the collaborating sites.  It is however, still higher than 2022 due to ProtoDUNE running and increased simulation for the far and near detectors. 

\begin{table}[ht]
\centering\csvautotabularright{external/DUNERSEUSAGE-2022-11-14.csv}
\caption{Summary  of DUNE disk areas known to rucio \cite{scotgrid}.  The CASTOR and FNAL Dcache areas are partially tape-backed and expandable. FNAL and CERN allocations are not provided by the reports but usage is.  }
\label{tab:RSEUsage}
\end{table}

\begin{figure}[h]
\centering\includegraphics[height=0.5\textwidth]{Parameters_2022-11-18-2026/Parameters_2022-11-18-2026-Cumulative-Disk.png}

\csvautotabularright{Parameters_2022-11-18-2026/Parameters_2022-11-18-2026-Cumulative-Disk.csv}\caption{Cumulative Disk needs in PB. Includes data lifetimes}\label{fig:Cumulative-Disk}
\end{figure}


\begin{table}[ht]
\centering\csvautotabularright{external/DiskResources-2021-2022-2023.csv}
\caption{Summary of disk pledges, allocations and usage for 2021-2022 with model request for 2023.  This is based on the 2022 CCB tables which are available in indico  \cite{CCB2022,CCB2023}.  These numbers are derived from the rucio reports in Table \ref{tab:RSEUsage} and may not be complete. }
\label{tab:DiskPledges}
\end{table}

\subsection{Tape}

DUNE currently has $\sim$23 PB of data on tape at Fermilab and 5 PB of protoDUNE data as a second copy at CERN.  The UK and the IN2P3 have made tape available but it has not yet been smoothly integrated into our data flow.  We will not be requesting additional tape space from the collaboration until we can use it efficiently. 

Figure and Table  \ref{fig:Cumulative-Tape}  summarize the cumulative disk projected by our model. These numbers are used to generate the requests for 2023. Table \ref{tab:CPUUsage} shows pledges and utilization for 2021-2022 and the request for 2023.

\begin{figure}[h]
\centering\includegraphics[height=0.4\textwidth]{Parameters_2022-11-18-2026/Parameters_2022-11-18-2026-Cumulative-Tape.png}

\csvautotabularright{Parameters_2022-11-18-2026/Parameters_2022-11-18-2026-Cumulative-Tape.csv}\caption{Cumulative Tape needs in PB. Includes data lifetimes}\label{fig:Cumulative-Tape}
\end{figure}

\section{CPU Needs}

CPU  slot time estimates are created by estimating the number of simulated and raw events taken and then scaling by the measured CPU time on a gpvm corrected for the estimated efficiency (default 70\%) and for a memory utilization factor that takes into account the differing memory needs for different applications and the number of slots needed to satisfy those needs.  Here we assume that reconstruction takes 4000MB, simulation takes 6000MB and actual measured  slot utilization for 2021 and 2022 is best fit by assuming that an average slot has 3000 MB of memory.    We assume that analysis takes the same interactive CPU time but uses only 3000 MB of memory.  

Figure/Table \ref{fig:CoresMain} shows the projected memory weighted wall-time projections through 2026.  This is different than in previous years where memory weighting was not applied. 

Table \ref{tab:CPUUsage} summarizes the pledges\cite{CCB2022} and measured usage using FNAL's statistics\cite{fifemonDUNE}.  The  usage numbers for 2022 are Nov 2021 to Oct 2022. 

Table \ref{tab:EIGSummary} summarizes the statistics for Nov 2021 to Oct 2022 derived from the alternate EIG accounting\cite{EIG2022}.  

\begin{figure}[h]
\centering\includegraphics[height=0.4\textwidth]{Parameters_2022-11-18-2026/Parameters_2022-11-18-2026-Cores.png}
\csvautotabularright{Parameters_2022-11-18-2026/Parameters_2022-11-18-2026-Cores.csv}
\caption{Slot weighted CPU needs in number of cores. Slot weighted wall time takes into account memory and efficiency.}\label{fig:CoresMain}
\end{figure}

\begin{table}[ht]
\centering\csvautotabularright{external/CPUresources-2021-2022-2023.csv}
\caption{Summary  of DUNE CPU pledges and contributions for 2021 and 2022.  Individual nations are listed and then merged (with US OSG) into a Collab section.  } \label{tab:CPUUsage}
\end{table}

\begin{table}[ht]
\centering\csvautotabularright{external/EIGSummary.csv}
\caption{Summary  of DUNE slot hours from European collaborators, Nov. 21 to Oct. 22, using the EIG accounting\cite{EIG2022}. These numbers differ slightly from the FNAL numbers in the previous table.} \label{tab:EIGSummary}
\end{table}


%\section{
%\section{Projected Disk and Tape needs by source and site}
\begin{figure}[h]
\centering\includegraphics[height=0.4\textwidth]{Parameters_2022-11-18-2026/Parameters_2022-11-18-2026-Cumulative-Tape.png}\label{fig:Cumulative-Tape}
\csvautotabularright{Parameters_2022-11-18-2026/Parameters_2022-11-18-2026-Cumulative-Tape.csv}\caption{Cumulative Tape needs in PB. Includes data lifetimes}
\end{figure}
\begin{figure}[h]
\centering\includegraphics[height=0.4\textwidth]{Parameters_2022-11-18-2026/Parameters_2022-11-18-2026-Cumulative-Disk.png}\label{fig:Cumulative-Disk}
\csvautotabularright{Parameters_2022-11-18-2026/Parameters_2022-11-18-2026-Cumulative-Disk.csv}\caption{Cumulative Disk needs in PB. Includes data lifetimes}
\end{figure}


%\section{Model Assumptions}

\cleardoublepage
%\renewcommand{\bibname}{References}
\bibliographystyle{utphys} 
\bibliography{bib}
\clearpage
\appendix

\section{Information about storage from SAM}\label{storage}

This section provides information on the sizes of data samples known to the SAM data catalog as of Nov. 1, 2022.  If a file has multiple copies, that is not shown here.  Tables \ref{tab:MCinSAM} and \ref{tab:DataInSam} show the total across all streams and data tiers while table \ref{tab:LargestSizes} shows the distribution of the largest samples.  


\begin{table}[ht]
 \centering\csvautotabularright{external/mc.csv}
\caption{Summary  of total simulation in SAM by detector type as of Nov 1, 2022.} 
\label{tab:MCinSAM}
\end{table}

\begin{table}[ht]
 \centering\csvautotabularright{external/detector.csv}
 \caption{Summary  of total detector data in SAM by detector type as of Nov 1, 2022.}
 \label{tab:DataInSam}
\end{table}



\begin{table}[ht]
 \centering\csvautotabularright{external/TOPTYPES.csv}
\caption{Classification of the largest data samples in SAM.  They are classified as detector(data) or mc, by the detector producing the data, by the stream (readout time) and by the data tier.  Some types, test and noise for example are archival only.  }
 \label{tab:LargestSizes}
\end{table}
\clearpage
\section{Model Details}

This appendix shows the parameters used in the model and plots of all the input and derived quantities as a function of time. 

Resource needs for reconstructed data for a given year are based on the number of events produced over the previous "Reprocess" years.   For ProtoDUNEs that is 2-4 years. 

Simulation resource needs are instead calculated based on a number of simulation events each year. The assumption is that new software versions imply resimulation.

Disk and tape lifetimes for different data types are specified as well as the desirable number of copies. 

The splits parameters make CERN responsible for raw data until 2027 with the collaboration taking over after that point. 

{\tt Detectors:} Detectors included in the calculation = {\tt ['SP', 'PDHD', 'DP', 'PDVD', 'HD', 'VD', 'ND-SAND', 'ND-LAr+TMS']} \\
{\tt Cap:} Cap on Raw data/year in PB = {\tt 30} \\
{\tt Base-Memory:} MB of memory per slot assumed as the average = {\tt 2000} \\
{\tt MaxYear:} Plot until year = {\tt 2030} \\
{\tt MinYear:} Plot starting with year = {\tt 2022} \\
{\tt Reprocess:} Number of years of data reprocessed when doing a new pass = {\tt {'SP': 3, 'DP': 3, 'PDHD': 3, 'PDVD': 3, 'PDs': 3, 'VD': 100, 'HD': 100, 'FDs': 100, 'ND-SAND': 100, 'ND-LAr+TMS': 100}} \\
{\tt AnalysisExtend:} Years analysis continues after last reco/sim = {\tt 3} \\
{\tt PatternFraction:} Fraction of time taken in pattern recognition = {\tt {'SP': 0.7, 'PDHD': 0.7, 'DP': 0.7, 'PDVD': 0.7, 'PDs': 0.7, 'HD': 0.1, 'VD': 0.1, 'FDs': 0.1, 'ND-SAND': 0.9, 'ND-LAr+TMS': 0.9, 'MARS': 0}} \\
{\tt TapeLifetimes:} Number of years kept on tape = {\tt {'Raw-Store': 100, 'Test': 1.0, 'Reco-Data-Store': 15, 'Sim-Store': 15}} \\
{\tt DiskLifetimes:} Number of years kept on disk = {\tt {'Raw-Store': 1, 'Test': 1.0, 'Reco-Data-Store': 3, 'Sim-Store': 2}} \\
{\tt TapeCopies:} Number of copies kept on tape = {\tt {'Raw-Store': 2, 'Test': 1, 'Reco-Data-Store': 1, 'Sim-Store': 1}} \\
{\tt DiskCopies:} Number of copies kept on disk = {\tt {'Raw-Store': 1, 'Test': 1, 'Reco-Data-Store': 2, 'Sim-Store': 1.5}} \\
{\tt PerYear:} Number of reprocessing done per year = {\tt {'Raw-Store': 1, 'Test': 1, 'Reco-Data-Store': 1, 'Sim-Store': 1, 'Events': 1, 'Sim-Events': 1, 'Reco-Data-CPU': 1, 'Sim-CPU': 1, 'Analysis': 1, 'Analysis-CPU': 1, 'Reco-Data-GPU': 1, 'Sim-GPU': 1}} \\
{\tt Cores:} Description of cores, efficiency and speed relative to 2020 vintage = {\tt {'Efficiency': 0.7, '2020Units': 1}} \\
{\tt kHEPSPEC06PerCPU:} kHEPSPEC06 per core assumed = {\tt 0.011} \\
{\tt SplitsYear:} Year CERN no longer responsible for disk or tape = {\tt 2029} \\
{\tt SplitsEarly:} Division between FNAL/CERN/Global for storage until SplitsYear = {\tt {'Tape': {'Raw-Store': {'FNAL': 0.5, 'CERN': 0.5, 'Global': 0.0}, 'Sim-Store': {'FNAL': 1.0, 'CERN': 0.0, 'Global': 0.0}, 'Reco-Data-Store': {'FNAL': 1.0, 'CERN': 0.0, 'Global': 0.0}, 'Test': {'FNAL': 0.5, 'CERN': 0.5, 'Global': 0.0}}, 'Disk': {'Raw-Store': {'FNAL': 0.5, 'CERN': 0.5, 'Global': 0.0}, 'Sim-Store': {'FNAL': 0.4, 'CERN': 0.1, 'Global': 0.5}, 'Reco-Data-Store': {'FNAL': 0.4, 'CERN': 0.1, 'Global': 0.5}, 'Test': {'FNAL': 0.5, 'CERN': 0.5, 'Global': 0.0}}, 'CPU': {'CPU': {'FNAL': 0.4, 'CERN': 0.1, 'Global': 0.5}}}} \\
{\tt SplitsLater:} Division between FNAL/CERN/Global for storage after SplitsYear = {\tt {'Tape': {'Raw-Store': {'FNAL': 0.5, 'CERN': 0.25, 'Global': 0.25}, 'Sim-Store': {'FNAL': 0.5, 'CERN': 0.0, 'Global': 0.5}, 'Reco-Data-Store': {'FNAL': 0.5, 'CERN': 0.0, 'Global': 0.5}, 'Test': {'FNAL': 0.5, 'CERN': 0.0, 'Global': 0.5}}, 'Disk': {'Raw-Store': {'FNAL': 1.0, 'CERN': 0.0, 'Global': 0.0}, 'Sim-Store': {'FNAL': 0.25, 'CERN': 0.0, 'Global': 0.75}, 'Reco-Data-Store': {'FNAL': 0.25, 'CERN': 0.0, 'Global': 0.75}, 'Test': {'FNAL': 0.5, 'CERN': 0.0, 'Global': 0.5}}, 'CPU': {'CPU': {'FNAL': 0.5, 'CERN': 0.0, 'Global': 0.5}}}} \\
{\tt filename:} Input configuration file = {\tt NearTerm\_2024-02-05-2040.json} \\

\end{document}

