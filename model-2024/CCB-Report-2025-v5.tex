\documentclass[12pt]{article}
\renewcommand{\familydefault}{\sfdefault}
\usepackage{csvsimple}
\usepackage{rotating}
\usepackage{graphicx}
\usepackage{hyperref}
\newcommand{\ignore}[1]{}
\parindent=0pt
\parskip=5 pt
\setlength{\textwidth=7in}
\setlength{\oddsidemargin=-.25 in}
\setlength{\topmargin=0 in}
\setlength{\textheight=8.5 in}
\input{NearTerm_2024-09-16-2030_noMWC/NearTerm_2024-09-16-2040macros.tex}
\makeatletter
\csvset{
  autotabularright/.style={
    file=#1,
    after head=\csv@pretable\begin{tabular}{|*{\csv@columncount}{r|}}\csv@tablehead,
    table head=\hline\csvlinetotablerow\\\hline,
    late after line=\\,
    table foot=\\\hline,
    late after last line=\csv@tablefoot\end{tabular}\csv@posttable,
    command=\csvlinetotablerow},
  autobooktabularright/.style={
    file=#1,
    after head=\csv@pretable\begin{tabular}{*{\csv@columncount}{r}}\csv@tablehead,
    table head=\toprule\csvlinetotablerow\\\midrule,
    late after line=\\,
    table foot=\\\bottomrule,
    late after last line=\csv@tablefoot\end{tabular}\csv@posttable,
    command=\csvlinetotablerow},
}
\makeatother

\newcommand{\csvautotabularright}[2][]{\csvloop{autotabularright={#2},#1}}
\newcommand{\csvautobooktabularright}[2][]{\csvloop{autobooktabularright={#2},#1}}

%\usepackage{draftwatermark}
%\SetWatermarkText{Draft}
%\SetWatermarkScale{5}

\title{DUNE Offline Computing Model Calculations for 2025}
\author{H. Schellman for the Computing Consortium}
\date{\today\ -- version 5}



\newcommand{\hrefII}[1]{\href{#1}{#1}}

%\input{NearTerm_2024-09-16-2030_noMWC/NearTerm_2024-09-16-2030_noMWC.tex}
\begin{document}


\makeatletter
%\csvset{
%  autotabularright/.style={
%    file=#1,
%    after head=\csv@pretable\begin{tabular}{|*{\csv@columncount}{r|}}\csv@tablehead,
%    table head=\hline\csvlinetotablerow\\\hline,
%    late after line=\\,
%    table foot=\\\hline,
%    late after last line=\csv@tablefoot\end{tabular}\csv@posttable,
%    command=\csvlinetotablerow},
%}
%\makeatother
%\newcommand{\csvautotabularright}[2][]{\csvloop{autotabularright={#2},#1}}

\maketitle

\tableofcontents
\newpage
\listoftables
\listoffigures

\section{Introduction}

This is a projection for DUNE CPU and storage needs intended for use at the Computing Contributions Board meeting in September 2024. It projects needs for \configRequestYear\ to \configMaxYear. 

The overall computing model  and 2023 projections for DUNE were described in chapters 6-13 of the  (Oct. 2022) DUNE Conceptual Design Report \cite{DUNE:2022fcw}.   This document provides updates on resource needs for 2024-2025. 

The \configRequestYear\ projection is done using codes at: \href{https://github.com/DUNE/CCB-data/tree/CCB-Jun24/models-2024/}{https://github.com/DUNE/CCB-data/tree/CCB-Jun24/models-2024/} from parameters stored in a json and csv file. We use CPU and storage sizes derived from protoDUNE and simulation experience and apply them to projected numbers of events from the various DUNE detectors. 

%This note summarizes the  projected need for the CCB.

%{\bf One Line Executive Summary:} Because the 2023 request assumed ProtoDUNE-2 running in 2023, the requests for 2024 are similar to the 2023 requests.





Changes since the last report \cite{CCB2024Report, CCB2024Minutes} include:

\begin{itemize}
\item A complete rewrite of the model code to make it more robust and allow more flexibility in plots. 
\item Delay in the start of ProtoDUNE-2 running at CERN until mid-2024. This moved the need for storage/CPU into mid 2024.  
\item More data from ProtoDUNE-HD than projected.  
\item Anticipate need to maintain both ProtoDUNE-HD and ProtoDUNE-VD raw data on disk simultaneously  for reprocessing.  The previous model assumed 1 year lifetimes for raw data on disk and this has been extended to \configRawDataDiskLifetimes\ years.  {\bf This leads to a significant increase in the 2025 need which may need to be mitigated by reducing disk access to legacy ProtoDUNE-SP/DP data.
See Figures  \ref{tab:CumulativeDiskByDetector2022-2030}, \ref{tab:CumulativeDiskByType2022-2030}, \ref{tab:CumulativeDiskBySite2022-2030} for projections.  }
\item This report now uses existing data catalog information up to August 2024 in forming a baseline for future projections. 
%\item Use of memory-weighted-core time instead of wall-time as our codes often require more memory than is available on a single core.  This means that our jobs sometimes need to reserve more than one core for a single process. This motivates the introduction of a memory-weighted wall-time unit for contributions  as different sites will  need to provide different amounts of wall-time to perform the same processing. 
\item Switch to HS23-Years as opposed to Core-Years as the main CPU reporting system.
\item Revisions to near-term requests based on the 2023 experience including a hold on tape requests from the collaboration during ProtoDUNE activities. 
%\item Simulation disk copies continue to be reduced from 2 to 1.5 to fit within a reasonable profile.  If more disk becomes available we can restore more simulation copies.  
\item Enhanced model for splits (Tables \ref{tab:DiskSplit}, \ref{tab:TapeSplit}, \ref{tab:CPUSplit}) between regions in resource provision. 
%\item A change in future tape requests to reflect the greater accessibility of tape archives at CERN and FNAL relative to other sites. 
\end{itemize}

\section{Model Summary}

Resource requests are based on a processing and storage model that separates different phases of the detectors and includes 5 classifications of activity.

\begin{itemize}
\item Physics Data - described by number of events, CPU time/event, storage/event.  Physics Data are described by number of events and then transformed into storage and CPU based on measured or estimated performance. 
\item Simulation - described by number of events,  CPU time/event, storage/event.  Currently both simulation and reconstruction are combined as a single entity as that is how we currently perform simulation.  The model will be extended to support multiple stages in future. 
\item Test - data taking as part of commissioning.  These have a short lifetime and are initially described by PB of storage. They are assumed not to consume large amounts of CPU and to only reside on disk briefly. 
\item Trigger Primitives (TP) -  these are currently described by PB of storage and are assumed not to consume large amounts of CPU.  
\item Analysis - Analysis is assumed to be use little disk but consume considerable CPU and IO resources.  It is currently described by a scaling factor ($\sim 50\%$) relative to Reconstruction and Simulation CPU.
\end{itemize}

 Table \ref{tab:detectors} summarizes the main detectors and their abbreviations. Smaller prototypes that generate less than 1 PB of data  are not explicitly included. 

\begin{table}[h]
\begin{centering}
  \begin{tabular}{|l|l|l|}
     \hline
    Abbrev. & Detector & Running time\\
    \hline
    SP & ProtoDUNE Single Phase & 2018-2021\\
    DP & ProtoDUNE Dual Phase & 2018-2021\\
    Coldbox & Both VD and HD & 2018-2024\\
    PDHD & ProtoDUNE-2 Horizonal Drift & 2024-2026\\
    PDVD & ProtoDUNE-2 Vertical Drift & 2024-2026\\
    2x2& Prototype NDLAr Detector & 2024-2026\\
    HD & Far Detector Horizontal Drift & 2028-\\
    VD & Far Detector Vertical Drift & 2028-\\
    NDLAr + TMS & Near Detector Liquid Argon + Muon System & 2030-\\
    SAND & On Axis Near Detector & 2030- \\
     \hline
     \end{tabular}
       \caption{Detector abbreviations and estimated running times.  Simulation campaigns occur earlier. }\label{tab:detectors}
  \end{centering}
   
     \end{table}
     
     

  These inputs are then used to calculate storage and CPU needs. 

\subsection{Storage replication and retention policies}
Disk storage policies are designed to optimize access and minimize CPU inefficiencies due to network transfer speeds.  In particular, all recent data samples are planned to have at least one disk copy to avoid the need to access tape.  As raw data processing is not I/O bound, one disk copy may be sufficient. Analysis of reconstructed samples is I/O bound so multiple copies located closer to CPU are desirable. 

Each type of data has a storage retention policy which includes lifetimes on disk and tape and number of copies on disk and tape.  For example, raw data has a very long tape retention policy and 2 tape copies, with a 2-3 year stay of 1 copy on disk.   Recent simulated and reconstructed data samples typically have 1 tape copy, 2 disk copies and disk retention times of 1.5-2 years to allow fast analysis.   An extended retention time after the end of data taking for the last version of sim/reco is added to allow for extended data analysis. 

Simulation and reconstruction from 2018 ProtoDUNE run is still active but is reduced to 1 disk copy. 

The model currently does not include any disk headroom allowance for data movement.  This will be added in the 2025 version of the model. 

\subsection{Processing calculations}

 We assume one processing campaign per year, with all real data reprocessed each campaign.   Simulation is assumed to be regenerated  once per year.
   
CPU calculations are now made in HS23 units.  Calculations for the ProtoDUNEs and DUNE far detector are based on existing reconstruction and simulation experience.  We have much less experience with Near Detector codes so those estimates are less accurate.

We assume ProtoDUNE running in 2024-2025 with startup of DUNE FD commissioning in 2027 (Test stream) and data taking in 2029. A cap of 30 PB/year for raw data (Events, Trigger Primitives and Tests) is imposed by scaling all categories down proportionally when 30 PB is reached. 

   
     
  \section{Details of the model}
  
  The model is implemented in python.  Time series are created for each combination of detector, data type, resource and location.  Those time series are then  scaled, extended and iterated and the results stored as new time series.  The end result is a csv file with a line for each permutation of types and figures summarizing various aggregations of data. Figure \ref{fig:Flowchart} shows the steps applied to data and simulation. 
  
  \begin{figure}[ht]
\centering
\includegraphics[width=0.99 \textwidth]{Flowchart.pdf}
\caption{Flow chart of transformations used to transform raw event counts into summaries such as cumulative disk on tape. }
\label{fig:Flowchart}
\end{figure} 
  
  \subsection{Inputs}
  
 Inputs to the model include projections of events and storage per year by detector and datatype, which are contained in a "timeline file".  A master configuration file in json format is used to store detector specific parameters which do not depend on year. 
 
 The parameters are stored in \href{https://github.com/DUNE/CCB-data/blob/CCB-Jun24/model-2024/NearTerm_2024-09-16-2040.json}{NearTerm\_2024-09-16-2040.json} and  \href{https://github.com/DUNE/CCB-data/blob/CCB-Jun24/model-2024/NearTerm_2024-09-16-2040_timeline.csv}{NearTerm\_2024-09-16-2040\_timeline.csv}
 
The calculation starts with numbers of events from the timeline\footnote{Event numbers for previous years are back-estimated from the total size of the data from the data catalog and an estimated size/event because the catalog does not report event number totals.  Early simulation stored very large events relative to the current estimated size, leading to an overestimate of the number of events but a correct estimate of disk and tape use.}  and uses those numbers, with CPU/storage times/event from Table \ref{tab:DetectorProcessingParameters} to calculate the annual new storage and CPU use for data taken in that year.  Figure/Tables \ref{tab:DataEventsPerYear2022-2030} and \ref{tab:SimEventsPerYear2022-2030} show the assumptions for numbers of events/year. 

{\input NearTerm_2024-09-16-2030_noMWC/Millions-of-Physics-Events-per-Year-by-Detector-Events.tex
}

{\input NearTerm_2024-09-16-2030_noMWC/Millions-of-Reconstructed-Simulated-Events-per-Year-by-Detector-Events.tex
}

Those data are then stored with sizes/event {\tt RawDataStore} and {\tt SimDataStore}.  See Table \ref{tab:DetectorProcessingParameters} for storage estimates per event by detector. 

\subsection{Simulation/Reconstruction Processing}

Processing is modeled by taking the number of events produced in a given year and multiplying by the measured CPU times {\tt RecoDataCPU} and {\tt SimDataCPU} for processing those events.  Those data are then assumed to be stored with size {\tt  RecoDataStore} and {\tt SimDataStore}.

Reprocessing for detector data is assumed to happen every year and cover {\tt Reprocess} = \configPDHDReprocess\ years for ProtoDUNE and the whole sample for the FD and ND.

New simulation is assumed to happen every year and previous years are not reprocessed. 

\begin{table}[h]
\centering
\footnotesize
\csvautotabularright{NearTerm_2024-09-16-2030_noMWC/NearTerm_2024-09-16-2040_detectors.csv}
\caption{Per event output size and CPU time  parameters used in the model. These values are estimated from running jobs over small numbers of events from protoDUNE and simulation.}
\label{tab:DetectorProcessingParameters}
\end{table}



\subsection{Copies and Retention}

Once the data volume generated per year is determined, those data are given retention times and a proposed number of copies. For popular datasets the number of disk copies should be two and disk copies should be retained for 2 years (two reconstruction/simulation cycles) if possible.  Due to the large volume of simulation and need for space for raw data from protoDUNE, the number of copies for simulation has been reduced to 1.5 to keep disk demands reasonable. 

Raw data has 2 copies on tape with a very long lifetime but a shorter disk lifetime as it is assumed that most users will use the reconstructed samples.

\begin{table}[h]
\centering
\footnotesize
\csvautotabularright{NearTerm_2024-09-16-2030_noMWC/NearTerm_2024-09-16-2040_diskcopies.csv}
\caption{Lifetimes and number of copies for different kinds of data.  An exception, we assume protoDUNE raw radata will stay on disk for up to three years for reprocessing.  Far detector data are assumed to stay for 2 years. }
\label{tab:Lifetimes}
\end{table}

\subsection{Analysis}

The Analysis model is currently crude.  Analysis processing is modeled by scaling the reconstruction and simulation CPU time by a multiplicative factor {\tt AnalysisCPU} $\sim 0.25-0.5 $ and extending the time by {\tt AnalysisExtend} = \configAnalysisExtend\ years.  Analysis use of reconstructed and simulated data is modeled by extending the disk lifetime of those data by the same amount.  Currently analysis data samples are assumed to be much smaller than the reconstructed and simulated samples. 
Figure/Table \ref{tab:CPU-kHS23-Yr-Types2022-2030} and \ref{tab:DataEventsPerYear2028-2038}. shows CPU usage by data type.

%\input NearTerm_2024-09-16-2030_noMWC/Processing-by-Type-CPU-kHS23-Yr.tex

 \subsection{Splitting resources across sites}
 Resources are assumed to be split as follows:
 
 \begin{itemize}
 \item Raw data for ProtoDUNE are stored at both CERN and Fermilab 
 \item Reconstructed and simulated data are split between the US (US) and the rest of the Collaboration (Global)
 \end{itemize}
 
 \begin{table}[h]
\centering
\footnotesize
\csvautotabularright{NearTerm_2024-09-16-2030_noMWC/NearTerm_2024-09-16-2040_splits_Disk.csv}
\caption{Assumptions about splits of Disk resources between the US, CERN and Global. }
\label{tab:DiskSplit}
\end{table}


\begin{table}[h]
\centering
\footnotesize
\csvautotabularright{NearTerm_2024-09-16-2030_noMWC/NearTerm_2024-09-16-2040_splits_Tape.csv}
\caption{Assumptions about splits of Tape resources between the US, CERN and Global.}
\label{tab:TapeSplit}
\end{table}

\begin{table}[h]
\centering
\footnotesize
\csvautotabularright{NearTerm_2024-09-16-2030_noMWC/NearTerm_2024-09-16-2040_splits_CPU.csv}
\caption{Assumptions about splits of CPU resources between the US, CERN and Global.}
\label{tab:CPUSplit}
\end{table}

\section{Resource use projections}

In this section we show projected resource needs by detector, data type, resource type and site.

\input NearTerm_2024-09-16-2030_noMWC/New-Disk-by-Detector-Storage.tex
\input NearTerm_2024-09-16-2030_noMWC/New-Disk-by-Type-Storage.tex
\input NearTerm_2024-09-16-2030_noMWC/New-Disk-by-Site-Storage.tex

\input NearTerm_2024-09-16-2030_noMWC/Cumulative-Disk-by-Detector-Storage.tex
\input NearTerm_2024-09-16-2030_noMWC/Cumulative-Disk-by-Type-Storage.tex
\input NearTerm_2024-09-16-2030_noMWC/Cumulative-Disk-by-Site-Storage.tex


\input NearTerm_2024-09-16-2030_noMWC/Cumulative-Tape-by-Detector-Storage.tex
\input NearTerm_2024-09-16-2030_noMWC/Cumulative-Tape-by-Type-Storage.tex
\input NearTerm_2024-09-16-2030_noMWC/Cumulative-Tape-by-Site-Storage.tex


\input NearTerm_2024-09-16-2030_noMWC/Processing-by-Detector-CPU-kHS23-Yr.tex
\input NearTerm_2024-09-16-2030_noMWC/Processing-by-Type-CPU-kHS23-Yr.tex
\input NearTerm_2024-09-16-2030_noMWC/CPU-processing-by-Site-CPU-kHS23-Yr.tex


\section{Longer term projections}

This section shows the projected resources past the startup of the far detectors in 2028-29. 

The estimates for storage are dominated by the far detector while CPU is likely dominated by the near detectors

In this section we show projected resource needs by detector, data type, resource type and site.

{\input NearTerm_2024-09-16-2038_noMWC/Millions-of-Physics-Events-per-Year-by-Detector-Events.tex
}

{\input NearTerm_2024-09-16-2038_noMWC/Millions-of-Reconstructed-Simulated-Events-per-Year-by-Detector-Events.tex
}


\input NearTerm_2024-09-16-2038_noMWC/New-Disk-by-Detector-Storage.tex
\input NearTerm_2024-09-16-2038_noMWC/New-Disk-by-Type-Storage.tex
\input NearTerm_2024-09-16-2038_noMWC/New-Disk-by-Site-Storage.tex


\input NearTerm_2024-09-16-2038_noMWC/Cumulative-Disk-by-Detector-Storage.tex
\input NearTerm_2024-09-16-2038_noMWC/Cumulative-Disk-by-Type-Storage.tex
\input NearTerm_2024-09-16-2038_noMWC/Cumulative-Disk-by-Site-Storage.tex

\input NearTerm_2024-09-16-2038_noMWC/Cumulative-Tape-by-Detector-Storage.tex
\input NearTerm_2024-09-16-2038_noMWC/Cumulative-Tape-by-Type-Storage.tex
\input NearTerm_2024-09-16-2038_noMWC/Cumulative-Tape-by-Site-Storage.tex


\input NearTerm_2024-09-16-2038_noMWC/Processing-by-Detector-CPU-kHS23-Yr.tex
\input NearTerm_2024-09-16-2038_noMWC/Processing-by-Type-CPU-kHS23-Yr.tex
\input NearTerm_2024-09-16-2038_noMWC/CPU-processing-by-Site-CPU-kHS23-Yr.tex


%%\end{document}
%
%
%
%\begin{table}[h]
%\begin{centering}
%%\caption{Division between FNAL/CERN/Global for storage until 2027}
%%{\bf Tape}
%%\begin{tabular}{|rrrr|}
%%\hline
%% &FNAL&CERN & Global \\
%% \hline
%%Raw:&  0.5&  0.5&  0.0\\
%%Sim:& 1.0&   0.0&   0.0\\ 
%%Reco-Data:& 1.0&   0.0&   0.0\\
%%Test: &  0.5&   0.5&  0.0\\
%%
%% \hline
%%  \end{tabular}
%
% %  {\bf Disk and CPU splits}
%     \begin{tabular}{|ll|rrr|}
%     \hline
% &&FNAL:&CERN & Global \\
% \hline
% {\bf Disk} &&&&\\
% &Raw&   0.50&   0.50&  0.00\\ 
% &Sim: & 0.40&  0.10&  0.50\\
% &Reco-Data: &  0.40&   0.10&  0.50\\ 
% &Test: &  0.50& 0.50&   0.00\\
%  \hline
%{\bf Tape} &&&&\\
%  &Raw:&   0.50&   0.50&  0.00\\ 
%  &Rest"&  1.00 & 0.00 & 0.00\\
%  \hline
%{\bf CPU} &&&&\\
%  &All: & 0.40& 0.10&0.50\\
%  \hline
%   \end{tabular}
%  \caption{Proposed division between FNAL/CERN/Global for storage and CPU in the near term, until $\sim$2028, when FD replaces ProtoDUNE as the primary source of experimental data.  The tape division is not yet finalized as we work on integration of Global tape archives. In the long run, Global sites are expected to take over some of the tape provision currently provided by CERN. }
%
%   \label{tab:division}
%   \end{centering}
%   \end{table}

%
%\section{Pledge Requests}
%
%Pledge requests for 2024 are similar to  those for 2023. Similar because  the impacts of ProtoDUNE running were already included in the 2023 request.
%
%These  pledge requests include only general resources available to the DUNE collaboration, not the additional local resources available to collaborators at their own institutions.  
%
%The near-term request does not include much Near Detector simulation.   That is currently being done via special allocations at NERSC and ANL and not yet included in our accounting.   
%
%The model for proposed pledges for \configRequestYear\ is detailed  in sections: Disk(\ref{sec:diskresult}), Tape(\ref{sec:taperesult}) and CPU(\ref{sec:cpuresult}) below.   
%\input{NearTerm_2024-09-16-2030_noMWC//Cumulative-Disk-by-Detector-Storage.tex}
%
%
%\ignore{A summary of requests and actual use for 2023 is shown in Table \ref{tab:summary2023} and the requests for \configRequestYear\ are shown in Table \ref{tab:summary2024} 
%
%\begin{table}[h]
%\begin{centering}
%
%\begin{tabular}{|ll|rr|rr|rr|}
%\hline
% 	&&	Disk (PB)	&	Actual	&	Tape(PB)	& Actual (PB)&CPU (Core-years)&Actual\\
%	\hline
%{\bf Model}	&&	25.80	&	--	&	45.5	&	-- & 15,169 & --	\\
%\hline
%{\bf Request}	&&		&		&		&	&	&\\
%&FNAL	&	7.80	&	9.80&	36.2	& 27.1&	3,792& 2,844	\\
%&CERN	&	2.60	&	4.02	&	9.2	& 5.7  &3,792	& 140\\
%&Global	&	15.40	&	9.76	&	0.1	&  0.1 &	7,585 &1, 524\\
%\hline
%&{\bf Total}	&	25.80	& 23.58		&	45.5	& 32.9 & 15,169 	 & 4,515\\
%\hline
%\end{tabular}
%
%\caption{Requests and actual usage from the previous year (2023).  CPU and disk utilization were low due to the delay in ProtoDUNE running. Table \ref{tab:RSEUsage}  shows the details of disk use by site.  CPU usage is detailed in  Table \ref{tab:CPUCores}. In this table  CPU is in units of unweighted core-years as that was the metric used prior to 2024.   }\label{tab:summary2023}
%\end{centering}
%
%\end{table}
%
%\begin{table}[h]
%\begin{centering}
%
%\begin{tabular}{|llr|r|r|r|}
%\hline
% 	&&	Disk (PB)	&		Tape(PB)	&	CPU (kHS23-years)	 & CPU (Core-years)\\
%	\hline
%{\bf Model}	&&	\DISKTotal	&		\TAPETotal	&	\CPUTotal	& \CORESTotal\\
%\hline
%{\bf Request}	&&		 		&		&		\\
%&US	&	\DISKFNAL&	 	\TAPEFNAL	&	\CPUFNAL & \CORESFNAL	\\
%&CERN	&	\configUSRawDataSdiskPDSplits	&	 	\TAPECERN&	\CPUCERN & \CORESCERN	\\
%&Global	&	\DISKGlobal	&	 	--	&	\CPUGlobal   &\CORESGlobal\\
%\hline
%&{\bf Total}	&	\DISKTotal	&	 	\TAPETotal	&	\CPUTotal	 & \CORESTotal\\
%\hline
%\end{tabular}
%
%\caption{Requests for 2024.  The disk requests reflect the different data types and the proposed splits from Table \ref{tab:division}.  They do not include the normal headroom of 5-10\%.  Tape pledges reflect the dominant use of CERN and FNAL for archival storage of data.  CPU pledges are in units of kHS23-years with Core-years provided for comparison to 2023.    }\label{tab:summary2024}
%\end{centering}
%
%\end{table}
%
%}
%
\clearpage

%\section{Supporting Materials}

%Generally, raw data from the protoDUNE detectors are stored on tape at both CERN and FNAL.  Simulation and reconstructed data  have one tape copy at Fermilab and recent reconstructed and simulated samples have one (or two) disk copies with one at Fermilab and one in Europe.  %Appendix \ref{storage} gives details on the size and types of data from the SAM data catalog.

%For the ProtoDUNE Runs CERN and FNAL have special responsibilities for archival data storage and for disk space for raw data while contributions from  other collaborating institutions are aggregated under the heading Global, which includes US sites outside of FNAL.  The proposed near-term  (until FD data taking) split between FNAL, CERN and Global contributions  is shown in Table \ref{tab:division}.  The pledges proposed here deviate slightly from those numbers with larger contributions from FNAL due to resources already in place. 
%

%\subsection{Disk}
%Table \ref{tab:RSEUsage} summarizes the disk utilization reported by sites via \cite{scotgrid}, supplemented by  rucio reports.   Some sites, notably TIFR, are not yet fully integrated so do not show up in the rucio reports.  %The contributions listed in Table \ref{tab:DiskPledges} sum the rucio and non-rucio disk known to be allocated to DUNE.
%
%\input{NearTerm_2024-09-16-2030_noMWC/Cumulative-Disk-by-Detector-Storage.tex}
%
%\end{document}
%Figure \ref{fig:Cumulative-Disk}  summarize the cumulative disk needs and requests projected by our model. These numbers are used to generate the request for \configRequestYear.  They are divided into the two host laboratories (CERN and FNAL) and Global, which includes contributions from the rest of the collaboration, including  BNL and NERSC in the US. 
%
%Table \ref{tab:summary2023} summarized the pledges from previous years compared to the actual amounts allocated shown in Table \ref{tab:GlobalUsage} .   Because of the delay in ProtoDUNE running, disk that had been planned for ProtoDUNE data and simulation was not fully used.  
%%\begin{table}[h]
%%\centering\csvautotabularright{external/DUNERSEUSAGE-2022-11-14.csv}
%%\caption{Summary  of DUNE disk areas known to rucio \cite{scotgrid}.  The CASTOR and FNAL Dcache areas are partially tape-backed and expandable. FNAL and CERN allocations are not provided by the reports but usage is.  }
%%\label{tab:RSEUsage}
%%\end{table}
%
%
%%\begin{table}[h]
%%%\begin{table}[h]
%%\centering
%%\csvautotabularright{external/StorageByCountry.csv}
%%\caption{Disk allocations and usage across countries at the end of 2023.    These numbers are derived from usage reports,  rucio reports and from cross-checks with individual sites on 2024-02-01.  The percentages are Used/Allocation. }
%%\label{tab:GlobalUsage}
%%%\end{table}
%%\end{table}
%
%\begin{sidewaystable}[h]
%%\begin{table}[h]
%\centering
%\csvautotabularright{external/RSEdata.csv}
%\caption{Disk allocations and usage across sites.    These numbers are derived from usage reports,  rucio reports and from cross-checks with individual sites on 2024-02-01.  The percentages are Used/Allocation. }
%\label{tab:RSEUsage}
%%\end{table}
%\end{sidewaystable}
%%\end{document}
%
%}
%\begin{figure}[h]
%\centering\includegraphics[height=0.5\textwidth]{NearTerm_2024-02-05-2030_noMWC/NearTerm_2024-02-05-2030_noMWC-Cumulative-Disk.png}
%\csvautotabularright{NearTerm_2024-02-05-2030_noMWC/NearTerm_2024-02-05-2030_noMWC-Cumulative-Disk-Source.csv}
%\csvautotabularright{NearTerm_2024-02-05-2030_noMWC/NearTerm_2024-02-05-2030_noMWC-Cumulative-Disk-Request.csv}
%\caption{Cumulative Disk needs in PB. Includes data lifetimes.  The top table shows the source of the data while the bottom table  shows the proposed split using the fractions from Table \ref{tab:division} and a modified version which reflects the disk already in place at FNAL and CERN, thus reducing the Global request. }\label{fig:Cumulative-Disk}
%\end{figure}
%
%
%
%
%%\begin{table}[h]
%%\centering\csvautotabularright{external/DiskResources-2021-2022-2023-2024-Details.csv}
%%\caption{Summary of disk pledges, allocations and usage for 2021-2022 with model request for 2023.  This is based on the 2022 CCB tables which are available in indico  \cite{CCB2022,CCB2023}.  These numbers are derived from the rucio reports in Table \ref{tab:RSEUsage} and may not be complete.  }
%%\label{tab:DiskPledges}
%%\end{table}
%
%\subsubsection{Request Summary}\label{sec:diskresult}
%The total disk request for 2024 is \DISKTotal PB.  This is consistent with the current allocated space of 24.3 PB. Because 5-10\% headroom is needed, we request that the current allocated  24.3 PB be retained for 2024. 
%
%\clearpage
%\subsection{Tape}
%
%\subsubsection{Current Status}
%
%DUNE currently has $\sim$27.1 PB of data\cite{fnaltape} on tape at Fermilab and 5.7 PB of raw protoDUNE data  at CERN\cite{scotgrid}. 
%
%\subsubsection{Model Projections}
%
%Figure  \ref{fig:Cumulative-Tape}  summarizes the cumulative  tape needs projected by  the model. These numbers are used to generate the requests for 2025.  They are divided into the two host laboratories (CERN and FNAL). The UK and the IN2P3 have offered $\sim 3$ PB of tape archive  but it has not yet been smoothly integrated into our data flow.  We  therefore substantially reduce our request for  resources  at those sites to $\sim 100$ TB per site, to allow testing of integration, with an increased request anticipated future years. .
%
%The increase in 2024-2025 is mainly storage of raw and processed  data from the upcoming runs of PDHD and PDVD.
%
%The exact division between FNAL, CERN and Global once DUNE starts taking FD data in 2029 is not yet defined.  
%
%
%
%\subsubsection{Current Status and Request}\label{sec:taperesult}
% We anticipate needing \TAPETotal\ PB of tape (an increase of 7 PB from 2023) to accommodate the ProtoDUNE run 2 data and increased simulation. 
%
%
%\begin{figure}[h]
%\centering\includegraphics[height=0.4\textwidth]{NearTerm_2024-02-05-2030_noMWC/NearTerm_2024-02-05-2030_noMWC-Cumulative-Tape.png}
%
%\csvautotabularright{NearTerm_2024-02-05-2030_noMWC/NearTerm_2024-02-05-2030_noMWC-Cumulative-Tape-Source.csv}
%\csvautotabularright{NearTerm_2024-02-05-2030_noMWC/NearTerm_2024-02-05-2030_noMWC-Cumulative-Tape-Request.csv}
%%\csvautotabularright{external/NearTerm_2024-02-05-2030_noMWC-Cumulative-Tape-Request.csv}
%\caption{Cumulative Tape needs from the model in PB, includes data lifetimes.  The top table shows the origin of the data while the bottom table  shows the proposed split.  Global contributions are set low in \configRequestYear\ and grow thereafter as more tape archives are integrated. The exact division between FNAL, CERN and Global once DUNE starts taking FD data in 2029 is not yet defined.  
% }\label{fig:Cumulative-Tape}
%\end{figure}
%
%
%
%\clearpage
%\subsection{CPU }
%
%\subsubsection{Current Status}
%Tables \ref{tab:CPUCores} and \ref{tab:CPUusage} show the CPU used by country  in calendar 2023.  Due to the delay in ProtoDUNE running CPU use was dominated by code development and analysis of the older ProtoDUNE data.   Total CPU use was substantially lower than projected.
%%Table \ref{tab:CPUUsage} shows pledges and utilization for 2021-2022 and the request for 2023.  
%
%%DUNE differs from other HEP experiments in frequently requiring more memory/core than is available at particular sites.  For example an 8-slot pilot with 16 GB of available memory may only accommodate four reconstruction processes.   As a result, we make our requests in terms of memory-weighted-core wall time (MWC)  with the base memory being 2000 MB. This maps reasonably well to the memory weighted slot-time returned by HTCondor and the slot-time reported in EGI statistics.  Sites that offer more  (or less) than 2000 MB/core can scale their contributions up by the local memory/core.
%
%\begin{table}[h]
%\centering\csvautotabularright{external/Usage_CoreYears_2023-01-01-2023-12-31_ByCountry.csv}
%\caption{CPU utilization in Core Years  for calendar 2023 divided by use case.  This is for comparison with the previous pledging system.    Production includes  official reconstruction and simulation. Analysis is user analysis of data.  MARS is beamline simulations performed at Fermilab.  NoMARS sums just Production and Analysis.  }
%\label{tab:CPUCores}
%\end{table}
%
%\begin{table}[h]
%\centering\csvautotabularright{external/Usage_kHS23-Yrs_2023-01-01-2023-12-31_ByCountry.csv}
%\caption{CPU utilization in kHS23-Years for calendar 2023 divided by use case.   Production includes official reconstruction and simulation. Analysis is user analysis of data.  MARS is beamline simulations performed at Fermilab.  NoMARS sums just Production and Analysis. }
%\label{tab:CPUusage}
%\end{table}
%
%
%\subsubsection{Model calculation}
%The wall-time estimates in the model are created by estimating the number of simulated and raw events taken and then scaling by the measured CPU time per event on a gpvm corrected to wall-time by the estimated efficiency (default 70\%).  Each data type also has an additional "Analysis" factor of 0.5 to 1.0  to account for analysis after the original reconstruction.  
%
%The numbers for PD and FD are based on substantial experience with the earlier ProtoDUNE runs and  large-scale simulation campaigns.  The numbers for ND are much more uncertain and await experience with  simulation and reconstruction of results from the 2x2 demonstrator which will run later in \configRequestYear\ at Fermilab. The model shows ProtoDUNE dominating through the 2024-2025 runs and subsequent analysis period with  testing and simulation for the DUNE near and far detectors taking over circa 2028. 
%
%
% %and for a memory utilization factor that takes into account the differing memory needs for different applications. Here we assume that analysis takes 3000 MB, reconstruction takes 4000MB, and simulation takes 6000MB.  Contributions are then requested in units of MWC-time which is wall-time$\times$2000 MB units. 
%
%
%%\begin{figure}[h]
%%\centering\includegraphics[height=0.4\textwidth]{NearTerm_2024-02-05-2030_noMWC/NearTerm_2024-02-05-2030_noMWC-HS23.png}
%%\csvautotabularright{NearTerm_2024-02-05-2030_noMWC/NearTerm_2024-02-05-2030_noMWC-HS23.csv}  %had to fix so moved to external
%%\caption{Proposed wall-time needs in number of 2000 MB HS23 (MWC-years). Memory-weighted  wall-time takes into account memory and efficiency.}\label{fig:HS23Main}
%%\end{figure}
%
%\begin{figure}[h]
%\centering\includegraphics[height=0.4\textwidth]{NearTerm_2024-02-05-2030_noMWC/NearTerm_2024-02-05-2030_noMWC-HS23.png}
%\csvautotabularright{NearTerm_2024-02-05-2030_noMWC/NearTerm_2024-02-05-2030_noMWC-HS23.csv}  %had to fix so moved to external
%\caption{Model of CPU wall-time needs  through 2030 in  kHS23-yr units.  The top 4 lines show division by detector, the bottom 4 show the proposed division between FNAL, CERN and Global based on the fractions in Table \ref{tab:division}.}\label{fig:HS23Main}
%\end{figure}
%
%
%Figure   \ref{fig:HS23Main} shows the projected wall-time  (HS23-yrs) need projections through 2030.   They are divided into the two host laboratories (CERN and FNAL) and Global, which includes contributions from the rest of the collaboration, including OSG, BNL and NERSC in the US. 
%
%
%%
%%\subsection{Example of memory weighted pledges}
%%An example of a  national pledge in MWC might be  1000 cores with 2GB available/core, 500 cores with 4 GB available/core and 100 cores with 8 GB available/core.  The MWC pledge would then be 
%%
%%\newcommand{\GB}{\hbox{GB}}
%%
%%\begin{eqnarray*} 1000\times2\GB/2\GB &+& \\ 500\times4\GB/2\GB &+&\\100\times8\GB/2\GB\\&&= 2400 \hbox{\ MWC units}\end{eqnarray*}.
%%
%%A pledge with cores with $<$ 2 GB would get partial MWC units per core. 
%%
%%The idea here is make the additional load of running large DUNE jobs transparent to sites, which either need to provide more than 2GB of memory/job or assign more cores than are actually used to a given job.  How a site makes and meets a pledge is up to the site management. Table \ref{tab:VOcard} summarizes the memory specifications from existing "VO" cards for the different experiments.  DUNE and LHCb currently are the only ones with a stated maximum $>$ 2048 MB.
%
%%\subsection{CPU Requests}
%%
%%We request \CPUTotal\  kHS23-Years of computing for 2024 to support reconstruction, calibration and analysis of ProtoDUNE data and simulation studies for the future far and near detectors. 
%%
%
%
%
%
%
%%Table \ref{tab:CPUUsage} summarizes previous pledges\cite{CCB2023}. %and the measured usage  for 2021 and 2022 using FNAL's  HTCondor memory-weighted wall-time statistics\cite{fifemonDUNE}.  The  usage numbers for 2022 are Nov 2021 to Oct 2022. 
%
%%Table \ref{tab:EIGSummary} summarizes the statistics for European sites from Nov 2021 to Oct 2022 derived from the EGI accounting\cite{EGI2022} which uses the number of cores allocated to a pilot.   If four 4000 MB reconstruction jobs were sent to an 8-core pilot on a system with 2000MB/core, this would be equivalent to using 8 MWC (memory-weighted wall-time units).   
%
%
%%\begin{table}[h]
%%\centering\csvautotabularright{external/CPUresources-2021-2022-2023-v3.csv}
%%\caption{Summary  of DUNE wall-time pledges and contributions for 2021 and 2022.  The 2022 actual numbers are memory-weighted core-years.  Individual nations are listed and then merged (with US OSG) into a Global section.} 
%%\label{tab:CPUUsage}
%%\end{table}
%
%%\begin{table}[h]
%%\centering\csvautotabularright{external/EIG-2022.csv}
%%\centering\csvautotabularright{external/EIG-2022-Global.csv}
%%\caption{Summary  of DUNE slot-years used for European collaborators, Nov. 21 to Oct. 22, using the EGI accounting\cite{EGI2022}.  The top shows all sites with EGI records, the bottom shows the summed contributions. These numbers reflect multiple-core slot reservations to account for larger memory use and are similar to, but  generally higher than, the FNAL accounting numbers in the previous table.} \label{tab:EIGSummary}
%%\end{table}
%%
%%\begin{table}[h]
%%\centering
%%\begin{tabular}{|l|l|l|}
%%\hline
%%Expt.&Max Memory& EGI VO Card\\
%%\hline
%%DUNE & 4000& \hrefII https://operations-portal.egi.eu/vo/view/voname/DUNE \\
%%ATLAS & 2000& \hrefII https://operations-portal.egi.eu/vo/view/voname/atlas \\
%%CMS & 2000& \hrefII https://operations-portal.egi.eu/vo/view/voname/cms \\
%%LHCb & 4000& \hrefII https://operations-portal.egi.eu/vo/view/voname/lhcb \\
%%ALICE& 2000& \hrefII https://operations-portal.egi.eu/vo/view/voname/alice \\
%%BelleII&2048& \hrefII https://operations-portal.egi.eu/vo/view/voname/belle \\
%%\hline
%%\end{tabular}
%%\caption{Maximum memory statements from the VO cards of major experiments.}
%%\label{tab:VOcard}
%%\end{table}
%%
%
%\subsubsection{CPU Request}\label{sec:cpuresult} The advent of ProtoDUNE-2 running in \configRequestYear\ and ramp-up of simulation for the FD and ND will lead to somewhat increased needs for CPU resources.  For \configRequestYear, we are requesting \CPUTotal\   kHS23-Yrs.
%%\section{
%%\section{Projected Disk and Tape needs by source and site}
\begin{figure}[h]
\centering\includegraphics[height=0.4\textwidth]{NearTerm_2024-02-05-2030_noMWC/NearTerm_2024-02-05-2030_noMWC-Cumulative-Tape.png}
\csvautotabularright{NearTerm_2024-02-05-2030_noMWC/NearTerm_2024-02-05-2030_noMWC-Cumulative-Tape.csv}\caption{Cumulative Tape needs in PB. Includes multiple copies and data lifetimes. The top 4 lines show the source of the data while the last four propose responsibilities.}
\label{fig:Cumulative-Tape}
\end{figure}
\begin{figure}[h]
\centering\includegraphics[height=0.4\textwidth]{NearTerm_2024-02-05-2030_noMWC/NearTerm_2024-02-05-2030_noMWC-Cumulative-Disk.png}
\csvautotabularright{NearTerm_2024-02-05-2030_noMWC/NearTerm_2024-02-05-2030_noMWC-Cumulative-Disk.csv}\caption{Cumulative Disk needs in PB. Includes multiple copies and data lifetimes. The top 4 lines show the source of the data while the last four propose responsibilities.}
\label{fig:Cumulative-Disk}
\end{figure}

%
%%\sectionHS23{Model Assumptions}

\cleardoublepage
%\renewcommand{\bibname}{References}
\bibliographystyle{utphys} 
\bibliography{bib}
\clearpage
\appendix

%\section{Information about storage from SAM}\label{storage}
%
%This section provides information on the sizes of data samples known to the SAM data catalog as of Nov. 1, 2022.  If a file has multiple copies, that is not shown here.  Tables \ref{tab:MCinSAM} and \ref{tab:DataInSam} show the total across all streams and data tiers while table \ref{tab:LargestSizes} shows the distribution of the largest samples.  
%
%
%\begin{table}[h]
% \centering\csvautotabularright{external/mc.csv}
%\caption{Summary  of total simulation in SAM by detector type as of Nov 1, 2022.} 
%\label{tab:MCinSAM}
%\end{table}
%
%\begin{table}[h]
% \centering\csvautotabularright{external/detector.csv}
% \caption{Summary  of total detector data in SAM by detector type as of Nov 1, 2022.}
% \label{tab:DataInSam}
%\end{table}
%
%
%
%\begin{table}[h]
% \centering\csvautotabularright{external/TOPTYPES.csv}
%\caption{Classification of the largest data samples in SAM.  They are classified as detector(data) or mc, by the detector producing the data, by the stream (readout time) and by the data tier.  Some types, test and noise for example are archival only.  }
% \label{tab:LargestSizes}
%\end{table}
%\clearpage
%\section{Model Details}
%
%This appendix shows the parameters used in the model and plots of all the input and derived quantities as a function of time. 
%
%Resource needs for reconstructed data for a given year are based on the number of events produced over the previous "Reprocess" years.   For ProtoDUNEs that is 2-4 years. 
%
%Simulation resource needs are instead calculated based on a number of simulation events each year. The assumption is that new software versions imply resimulation.
%
%Disk and tape lifetimes for different data types are specified as well as the desirable number of copies. 
%
%The splits parameters make CERN responsible for raw data until 2027 with the collaboration taking over after that point. 
%
%{\tt Detectors:} Detectors included in the calculation = {\tt ['SP', 'PDHD', 'DP', 'PDVD', 'HD', 'VD', 'ND-SAND', 'ND-LAr+TMS']} \\
{\tt Cap:} Cap on Raw data/year in PB = {\tt 30} \\
{\tt Base-Memory:} MB of memory per slot assumed as the average = {\tt 2000} \\
{\tt MaxYear:} Plot until year = {\tt 2030} \\
{\tt MinYear:} Plot starting with year = {\tt 2022} \\
{\tt Reprocess:} Number of years of data reprocessed when doing a new pass = {\tt {'SP': 3, 'DP': 3, 'PDHD': 3, 'PDVD': 3, 'PDs': 3, 'VD': 100, 'HD': 100, 'FDs': 100, 'ND-SAND': 100, 'ND-LAr+TMS': 100}} \\
{\tt AnalysisExtend:} Years analysis continues after last reco/sim = {\tt 3} \\
{\tt PatternFraction:} Fraction of time taken in pattern recognition = {\tt {'SP': 0.7, 'PDHD': 0.7, 'DP': 0.7, 'PDVD': 0.7, 'PDs': 0.7, 'HD': 0.1, 'VD': 0.1, 'FDs': 0.1, 'ND-SAND': 0.9, 'ND-LAr+TMS': 0.9, 'MARS': 0}} \\
{\tt TapeLifetimes:} Number of years kept on tape = {\tt {'Raw-Store': 100, 'Test': 1.0, 'Reco-Data-Store': 15, 'Sim-Store': 15}} \\
{\tt DiskLifetimes:} Number of years kept on disk = {\tt {'Raw-Store': 1, 'Test': 1.0, 'Reco-Data-Store': 3, 'Sim-Store': 2}} \\
{\tt TapeCopies:} Number of copies kept on tape = {\tt {'Raw-Store': 2, 'Test': 1, 'Reco-Data-Store': 1, 'Sim-Store': 1}} \\
{\tt DiskCopies:} Number of copies kept on disk = {\tt {'Raw-Store': 1, 'Test': 1, 'Reco-Data-Store': 2, 'Sim-Store': 1.5}} \\
{\tt PerYear:} Number of reprocessing done per year = {\tt {'Raw-Store': 1, 'Test': 1, 'Reco-Data-Store': 1, 'Sim-Store': 1, 'Events': 1, 'Sim-Events': 1, 'Reco-Data-CPU': 1, 'Sim-CPU': 1, 'Analysis': 1, 'Analysis-CPU': 1, 'Reco-Data-GPU': 1, 'Sim-GPU': 1}} \\
{\tt Cores:} Description of cores, efficiency and speed relative to 2020 vintage = {\tt {'Efficiency': 0.7, '2020Units': 1}} \\
{\tt kHEPSPEC06PerCPU:} kHEPSPEC06 per core assumed = {\tt 0.011} \\
{\tt SplitsYear:} Year CERN no longer responsible for disk or tape = {\tt 2029} \\
{\tt SplitsEarly:} Division between FNAL/CERN/Global for storage until SplitsYear = {\tt {'Tape': {'Raw-Store': {'FNAL': 0.5, 'CERN': 0.5, 'Global': 0.0}, 'Sim-Store': {'FNAL': 1.0, 'CERN': 0.0, 'Global': 0.0}, 'Reco-Data-Store': {'FNAL': 1.0, 'CERN': 0.0, 'Global': 0.0}, 'Test': {'FNAL': 0.5, 'CERN': 0.5, 'Global': 0.0}}, 'Disk': {'Raw-Store': {'FNAL': 0.5, 'CERN': 0.5, 'Global': 0.0}, 'Sim-Store': {'FNAL': 0.4, 'CERN': 0.1, 'Global': 0.5}, 'Reco-Data-Store': {'FNAL': 0.4, 'CERN': 0.1, 'Global': 0.5}, 'Test': {'FNAL': 0.5, 'CERN': 0.5, 'Global': 0.0}}, 'CPU': {'CPU': {'FNAL': 0.4, 'CERN': 0.1, 'Global': 0.5}}}} \\
{\tt SplitsLater:} Division between FNAL/CERN/Global for storage after SplitsYear = {\tt {'Tape': {'Raw-Store': {'FNAL': 0.5, 'CERN': 0.25, 'Global': 0.25}, 'Sim-Store': {'FNAL': 0.5, 'CERN': 0.0, 'Global': 0.5}, 'Reco-Data-Store': {'FNAL': 0.5, 'CERN': 0.0, 'Global': 0.5}, 'Test': {'FNAL': 0.5, 'CERN': 0.0, 'Global': 0.5}}, 'Disk': {'Raw-Store': {'FNAL': 1.0, 'CERN': 0.0, 'Global': 0.0}, 'Sim-Store': {'FNAL': 0.25, 'CERN': 0.0, 'Global': 0.75}, 'Reco-Data-Store': {'FNAL': 0.25, 'CERN': 0.0, 'Global': 0.75}, 'Test': {'FNAL': 0.5, 'CERN': 0.0, 'Global': 0.5}}, 'CPU': {'CPU': {'FNAL': 0.5, 'CERN': 0.0, 'Global': 0.5}}}} \\
{\tt filename:} Input configuration file = {\tt NearTerm\_2024-02-05-2040.json} \\


\end{document}
%\end{document}
%https://tex.stackexchange.com/questions/292512/csvsimple-csvautotabular-and-csvautobooktabular-with-centered-columns-content
