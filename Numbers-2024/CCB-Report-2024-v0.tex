\documentclass[12pt]{article}
\renewcommand{\familydefault}{\sfdefault}
\usepackage{csvsimple}
\usepackage{rotating}
\usepackage{graphicx}
\usepackage{hyperref}
\parindent=0pt
\parskip=5 pt
\setlength{\textwidth=7in}
\setlength{\oddsidemargin=-.25 in}
\setlength{\topmargin=0 in}
\setlength{\textheight=8.5 in}
%\usepackage{draftwatermark}
%\SetWatermarkText{Draft}
%\SetWatermarkScale{5}
\title{DUNE Offline Computing Model Calculations for 2024}
\author{H. Schellman for the Computing Consortium}
\date{\today -- version 0}



\newcommand{\hrefII}[1]{\href{#1}{#1}}
\begin{document}


\makeatletter
\csvset{
  autotabularright/.style={
    file=#1,
    after head=\csv@pretable\begin{tabular}{|*{\csv@columncount}{r|}}\csv@tablehead,
    table head=\hline\csvlinetotablerow\\\hline,
    late after line=\\,
    table foot=\\\hline,
    late after last line=\csv@tablefoot\end{tabular}\csv@posttable,
    command=\csvlinetotablerow},
}
\makeatother
\newcommand{\csvautotabularright}[2][]{\csvloop{autotabularright={#2},#1}}

\maketitle



\section{Introduction}

This is an annual projection for DUNE CPU and storage needs intended for use at the Computing Contributions Board meeting in February 2024. It projects needs for 2024 to 2030. 

The overall computing model  and 2023 projections for DUNE were described in chapters 6-13 of the recent (Oct. 2022) DUNE Conceptual Design Report \cite{DUNE:2022fcw}.   This document provides updates on resource needs for 2024-2025. 

The 2024 projection is done using codes at: \href{https://github.com/DUNE/CCB-data/tree/CCB-Feb24/Numbers-2024/}{https://github.com/DUNE/CCB-data/tree/CCB-Feb24/Numbers-2024/} from parameters stored in a json and csv file. We use CPU and storage sizes derived from protoDUNE and simulation experience and apply them to projected numbers of events from the various DUNE detectors. 

Details are provided in the appendices while the main body of this note summarizes pledges, usage and projected need for the CCB.





Changes since the last report \cite{CCBReport2023} include:

\begin{itemize}
\item A later start for ProtoDUNE-2 running at CERN. This leads needs for storage/CPU in mid 2024. As the 2023 estimates were made assuming a 2023 run, there is little change necessary in the request for 2024. 

%\item Use of memory-weighted-core time instead of wall-time as our codes often require more memory than is available on a single core.  This means that our jobs sometimes need to reserve more than one core for a single process. This motivates the introduction of a memory-weighted wall-time unit for contributions  as different sites will  need to provide different amounts of wall-time to perform the same processing. 
\item Removal of the memory weighted core concept as it caused too much confusion and most available cores now have more than 2000 MB available. 
\item Revisions to near-term requests based on the 2023 experience including a hold on tape requests from the collaboration during protoDUNE activities. 
\item Simulation disk copies continue to be reduced from 2 to 1.5 to fit within a reasonable profile.  If more disk becomes available we can restore more simulation copies.  
%\item A change in future tape requests to reflect the greater accessibility of tape archives at CERN and FNAL relative to other sites. 
\end{itemize}

\section{Model Summary}

Resource requests are based on a processing and storage model that separated different phases of the detectors and includes 4 classifications of activity.

\begin{itemize}
\item Data - described by number of events, CPU time/event, storage/event
\item Simulation - described by number of events,  CPU time/event, storage/event
\item Test - described by PB of storage, assumed not to consume large amounts of CPU
\item Analysis - described by a scaling factor relative to Reconstruction and Simulation CPU.
\end{itemize}

Those inputs are then used to calculate storage and CPU needs. 

Disk storage policies are designed to optimize access and minimize inefficiencies due to network transfer speeds.  As raw data processing is not I/O bound, one disk copy has been found to be sufficient.  Analysis of reconstructed samples is I/O bound so multiple copies located closer to CPU are desirable. 

Each type of data has a storage retention policy which includes lifetimes on disk and tape and number of copies on disk and tape.  For example, raw data has a very long tape retention policy and 2 tape copies, with a 1 year stay of 1 copy on disk.   Recent simulated and reconstructed data samples typically have 1 tape copy, 2 disk copies and disk retention times of 1.5-2 years to allow fast analysis.   We assume one processing campaign per year, with all real data reprocessed each campaign.   Simulation is assumed to be redone as well but older samples are not reprocessed.  An extended retention time after the end of data taking for the last version of sim/reco is added to allow for extended data analysis. 

CPU calculations are now made in HS23 units, with results from 2022 and 2023 included for reference.  Calculations for the protoDUNE's and DUNE far detector are based on existing reconstruction and simulation experience.  We have much less experience with Near Detector codes so those estimates are less accurate.

We assume protoDUNE running in 2024-2025 with startup of DUNE FD commissioning in 2027 (Test stream) and data taking in 2029. 

\section{Pledge Proposals}

Proposed pledges for 2024 are detailed  in sections: Disk(\ref{sec:diskresult}), Tape(\ref{sec:taperesult}) and CPU(\ref{sec:cpuresult} )below.   A summary of requests for 2023 is shown here in Table \ref{tab:summary2023} and the requests for 2024 are shown in \ref{tab:summary2024} 

\begin{table}[ht]
\begin{centering}

\begin{tabular}{|ll|rr|r|r|}
\hline
 	&&	Disk (PB)	&	Modified Disk (PB)	&	Tape(PB)	&	CPU (MWC-years)	\\
	\hline
{\bf Model}	&&	25.80	&	25.80	&	45.5	&	15,169	\\
\hline
{\bf Request}	&&		&		&		&		\\
&FNAL	&	7.80	&	8.86	&	36.2	&	3,792	\\
&CERN	&	2.60	&	4.00	&	9.2	&	3,792	\\
&Global	&	15.40	&	12.94	&	0.1	&	7,585\\
\hline
&{\bf Total}	&	25.80	&	25.80	&	45.5	&	15,169	\\
\hline
\end{tabular}

\caption{Requests from previous year (2023).  Disk pledges are based on existing CERN and FNAL contributions with Global contributions making up the rest of the model request.  Tape pledges reflect the dominant use of CERN and FNAL for archival storage of data.  CPU pledges are in units of memory-weighted-core-years and assume Fermilab and CERN each pledge 25\%.   }
\end{centering}
\label{tab:summary2023}
\end{table}

\begin{table}[ht]
\begin{centering}

\begin{tabular}{|ll|rr|r|r|}
\hline
 	&&	Disk (PB)	&		Tape(PB)	&	CPU (kHS23-years)	\\
	\hline
{\bf Model}	&&	24.1	&		45.5	&	15,169	\\
\hline
{\bf Request}	&&		 		&		&		\\
&FNAL	&	10.2	&	 	36.2	&	3,792	\\
&CERN	&	4.9	&	 	16.3	&	3,792	\\
&Global	&	9.0	&	 &	0.1	&	7,585\\
\hline
&{\bf Total}	&	24.1	&	 	44.7	&	15,169	\\
\hline
\end{tabular}

\caption{Requests for 2024).  Disk requests are based on existing CERN and FNAL contributions with Global contributions making up the rest of the model request.  Tape pledges reflect the dominant use of CERN and FNAL for archival storage of data.  CPU pledges are in units of memory-weighted-core-years and assume Fermilab and CERN each pledge 25\%.   }
\end{centering}
\label{tab:summary2024}
\end{table}



\clearpage

\section{Disk and Tape}

Generally, raw data are stored on tape at both CERN and FNAL.  Simulation and reconstructed data  have one tape copy at Fermilab and recent reconstructed and simulated samples have one (or two) disk copies with one at Fermilab and one in Europe.  Appendix \ref{storage} gives details on the size and types of data from the SAM data catalog.

CERN and FNAL have special responsibilities for archival data storage and for disk space for raw data while contributions from  other collaborating institutions are aggregated under the heading Global, which includes US sites outside of FNAL.  The traditional split between FNAL, CERN and Global contributions until 2028 is shown in Table \ref{tab:division}.  The pledges proposed here deviate slightly from those numbers with larger contributions from the host laboratories in 2023 due to resources already in place. 

\begin{table}[h]
\begin{centering}
%\caption{Division between FNAL/CERN/Global for storage until 2027}
%{\bf Tape}
%\begin{tabular}{|rrrr|}
%\hline
% &FNAL&CERN & Global \\
% \hline
%Raw:&  0.5&  0.5&  0.0\\
%Sim:& 1.0&   0.0&   0.0\\ 
%Reco-Data:& 1.0&   0.0&   0.0\\
%Test: &  0.5&   0.5&  0.0\\
%
% \hline
%  \end{tabular}

   {\bf Disk}
     \begin{tabular}{|rrrr|}
     \hline
 &FNAL: 0.5&CERN & Global \\
 \hline
 Raw&   0.50&   0.50&  0.00\\ 
 Sim: & 0.25&  0.0&  0.75\\
  Reco-Data: &  0.25&   0.00&  0.75\\ 
  Test: &  0.50& 0.50&   0.00\\
  \hline
   \end{tabular}
  \caption{Proposed division between FNAL/CERN/Global for storage until 2028.  The tape division is not yet finalized as we work on integration of Global tape archives. In the long run, Global sites are expected to take over some of the tape provision currently provided by CERN. }

   \label{tab:division}
   \end{centering}
   \end{table}

\subsection{Disk}
Table \ref{tab:RSEUsage} summarizes the disk utilization reported by sites via \cite{scotgrid}, supplemented by  rucio reports.   Some sites, notably TIFR, are not yet fully integrated so do not show up in the rucio reports.  The contributions listed in Table \ref{tab:DiskPledges} sum the rucio and non-rucio disk known to be allocated to DUNE.

Figure \ref{fig:Cumulative-Disk}  summarize the cumulative disk needs and requests projected by our model. These numbers are used to generate the request for 2023.  They are divided into the two host laboratories (CERN and FNAL) and Global, which includes contributions from the rest of the collaboration, including OSG, BNL and NERSC in the US. 

Table \ref{tab:DiskPledges} summarizes the pledges from previous years compared to the actual amounts allocated and used from Table \ref{tab:RSEUsage} .   The 2023 request has been re-evaluated in light of underuse in 2021 and 2022 and should better match the likely capacity of the collaborating sites.  It is however, still higher than 2022 due to ProtoDUNE running and increased simulation for the far and near detectors. 

%\begin{table}[ht]
%\centering\csvautotabularright{external/DUNERSEUSAGE-2022-11-14.csv}
%\caption{Summary  of DUNE disk areas known to rucio \cite{scotgrid}.  The CASTOR and FNAL Dcache areas are partially tape-backed and expandable. FNAL and CERN allocations are not provided by the reports but usage is.  }
%\label{tab:RSEUsage}
%\end{table}
\begin{sidewaystable}[ht]
%\begin{table}[ht]
\centering
\csvautotabularright{external/RSEdata.csv}
\caption{Disk allocations and usage across sites.    These numbers are derived from usage reports,  rucio reports and from cross-checks with individual sites on 2024-02-01.  The percentages are Used/Allocation. }
\label{tab:RSEUsage}
%\end{table}
\end{sidewaystable}

\begin{table}[ht]
%\begin{table}[ht]
\centering
\csvautotabularright{external/StorageByCountry.csv}
\caption{Disk allocations and usage across countries.    These numbers are derived from usage reports,  rucio reports and from cross-checks with individual sites on 2024-02-01.  The percentages are Used/Allocation. }
\label{tab:GlobalUsage}
%\end{table}
\end{table}


%\end{document}

\begin{figure}[h]
\centering\includegraphics[height=0.5\textwidth]{NearTerm_2024-02-02-2030_noMWC/NearTerm_2024-02-02-2030_noMWC-Cumulative-Disk.png}
\csvautotabularright{NearTerm_2024-02-02-2030_noMWC/NearTerm_2024-02-02-2030_noMWC-Cumulative-Disk-Source.csv}
\csvautotabularright{NearTerm_2024-02-02-2030_noMWC/NearTerm_2024-02-02-2030_noMWC-Cumulative-Disk-Request.csv}
\caption{Cumulative Disk needs in PB. Includes data lifetimes.  The top table shows the source of the data while the bottom table  shows the proposed split using the fractions from Table \ref{tab:division} and a modified version which reflects the disk already in place at FNAL and CERN, thus reducing the Global request. }\label{fig:Cumulative-Disk}
\end{figure}




\begin{table}[ht]
\centering\csvautotabularright{external/DiskResources-2021-2022-2023-2024-Details.csv}
\caption{Summary of disk pledges, allocations and usage for 2021-2022 with model request for 2023.  This is based on the 2022 CCB tables which are available in indico  \cite{CCB2022,CCB2023}.  These numbers are derived from the rucio reports in Table \ref{tab:RSEUsage} and may not be complete.  }
\label{tab:DiskPledges}
\end{table}

\subsubsection{Conclusion}\label{sec:diskresult}
The overall request for 2023 is 25.8 PB vs. the 22.6 PB already on the floor so we need to find 3.2 PB or descope the number of copies on disk. Currently CERN and FNAL contribute more  (12.9 PB on the floor vs. 9.1 PB request) than they would under the current divisions in Table \ref{tab:division} while national contributions are currently 9.7 PB vs 15.4 in the request based on the allocations in Table \ref{tab:division}.   A suggestion is to ask the collaborating institutions for the extra  3.2 PB of additional disk instead of the 5.7 PB that the current division of responsibility would suggest.  This is shown in Table \ref{tab:DiskPledges} as a modified request.


\clearpage
\subsection{Tape}



Figure  \ref{fig:Cumulative-Tape}  summarizes the cumulative  tape need projected by our model. These numbers are used to generate the requests for 2024.  They are divided into the two host laboratories (CERN and FNAL) and Global, which includes contributions from the rest of the collaboration, including OSG, BNL and NERSC in the US. 


\subsubsection{Conclusion}\label{sec:taperesult}
DUNE currently has $\sim$24.3 PB of data on tape at Fermilab and 5.7 PB of protoDUNE data as a second copy at CERN.  We anticipate needing up to 45 PB of tape (an increase of 16 PB from 2023) to accommodate the ProtoDUNE run 2 data and increased simulation. 

The UK and the IN2P3 have made $\sim 3$ PB of tape archive available but it has not yet been smoothly integrated into our data flow.  We  therefore substantially reduce our request for these resources to $\sim 100$ TB per site, to allow testing of integration, with an increased request anticipated future years. 

\begin{figure}[h]
\centering\includegraphics[height=0.4\textwidth]{NearTerm_2024-02-02-2030_noMWC/NearTerm_2024-02-02-2030_noMWC-Cumulative-Tape.png}

\csvautotabularright{NearTerm_2024-02-02-2030_noMWC/NearTerm_2024-02-02-2030_noMWC-Cumulative-Tape-Source.csv}
%\csvautotabularright{external/NearTerm_2024-02-02-2030_noMWC-Cumulative-Tape-Request.csv}
\caption{Cumulative Tape requests in PB, includes data lifetimes.  The top table shows the source of the data while the bottom table  shows the proposed split.  Global contributions are set low in 2023 and grow thereafter as more tape archives are integrated. }\label{fig:Cumulative-Tape}
\end{figure}



\clearpage
\section{CPU Needs}

%Table \ref{tab:CPUUsage} shows pledges and utilization for 2021-2022 and the request for 2023.  

%DUNE differs from other HEP experiments in frequently requiring more memory/core than is available at particular sites.  For example an 8-slot pilot with 16 GB of available memory may only accommodate four reconstruction processes.   As a result, we make our requests in terms of memory-weighted-core wall time (MWC)  with the base memory being 2000 MB. This maps reasonably well to the memory weighted slot-time returned by HTCondor and the slot-time reported in EGI statistics.  Sites that offer more  (or less) than 2000 MB/core can scale their contributions up by the local memory/core.

\subsection{Model calculation}
The wall-time estimates in the model are created by estimating the number of simulated and raw events taken and then scaling by the measured CPU time per event on a gpvm corrected to wall-time by the estimated efficiency (default 70\%).
The numbers for PD and FD are based on substantial experience with large-scale simulation campaigns.  The numbers for ND are much more uncertain and await experience with  simulation and reconstruction of results from the 2x2 demonstrator which will run later in 2024 at Fermilab. 


 %and for a memory utilization factor that takes into account the differing memory needs for different applications. Here we assume that analysis takes 3000 MB, reconstruction takes 4000MB, and simulation takes 6000MB.  Contributions are then requested in units of MWC-time which is wall-time$\times$2000 MB units. 


%\begin{figure}[h]
%\centering\includegraphics[height=0.4\textwidth]{NearTerm_2024-02-02-2030_noMWC/NearTerm_2024-02-02-2030_noMWC-HS23.png}
%\csvautotabularright{NearTerm_2024-02-02-2030_noMWC/NearTerm_2024-02-02-2030_noMWC-HS23.csv}  %had to fix so moved to external
%\caption{Proposed wall-time needs in number of 2000 MB HS23 (MWC-years). Memory-weighted  wall-time takes into account memory and efficiency.}\label{fig:HS23Main}
%\end{figure}

\begin{figure}[h]
\centering\includegraphics[height=0.4\textwidth]{NearTerm_2024-02-02-2030_noMWC/NearTerm_2024-02-02-2030_noMWC-HS23.png}
\csvautotabularright{NearTerm_2024-02-02-2030_noMWC/NearTerm_2024-02-02-2030_noMWC-HS23.csv}  %had to fix so moved to external
\caption{Proposed wall-time needs in  HS23 units. }\label{fig:HS23Main}
\end{figure}


Figure/Table \ref{fig:HS23Main} shows the projected wall-time  (HS23-yrs) need projections through 2030.   They are divided into the two host laboratories (CERN and FNAL) and Global, which includes contributions from the rest of the collaboration, including OSG, BNL and NERSC in the US. 

\begin{table}[ht]
\centering\csvautotabularright{external/Usage_kHS23-Yrs_2022-01-01-2022-12-31_ByCountry.csv}
\caption{CPU utilization in kHS23-Years for calendar 2022 divided by use case.   Production includes all offical reconstruction and simulation. Analysis is user analysis of data.  MARS is beamline simulations performed at Fermilab.  No-Mars sums Production and Analysis only.  }
\label{tab:DiskPledges}
\end{table}

\begin{table}[ht]
\centering\csvautotabularright{external/Usage_kHS23-Yrs_2023-01-01-2023-12-31_ByCountry.csv}
\caption{CPU utilization in kHS23-Years for calendar 2023 divided by use case.   Production includes all offical reconstruction and simulation. Analysis is user analysis of data.  MARS is beamline simulations performed at Fermilab.  No-Mars sums Production and Analysis only.  }
\label{tab:DiskPledges}
\end{table}

%
%\subsection{Example of memory weighted pledges}
%An example of a  national pledge in MWC might be  1000 cores with 2GB available/core, 500 cores with 4 GB available/core and 100 cores with 8 GB available/core.  The MWC pledge would then be 
%
%\newcommand{\GB}{\hbox{GB}}
%
%\begin{eqnarray*} 1000\times2\GB/2\GB &+& \\ 500\times4\GB/2\GB &+&\\100\times8\GB/2\GB\\&&= 2400 \hbox{\ MWC units}\end{eqnarray*}.
%
%A pledge with cores with $<$ 2 GB would get partial MWC units per core. 
%
%The idea here is make the additional load of running large DUNE jobs transparent to sites, which either need to provide more than 2GB of memory/job or assign more cores than are actually used to a given job.  How a site makes and meets a pledge is up to the site management. Table \ref{tab:VOcard} summarizes the memory specifications from existing "VO" cards for the different experiments.  DUNE and LHCb currently are the only ones with a stated maximum $>$ 2048 MB.

\subsection{Requests}




Table \ref{tab:CPUUsage} summarizes previous pledges\cite{CCB2022} and the measured usage  for 2021 and 2022 using FNAL's  HTCondor memory-weighted wall-time statistics\cite{fifemonDUNE}.  The  usage numbers for 2022 are Nov 2021 to Oct 2022. 

%Table \ref{tab:EIGSummary} summarizes the statistics for European sites from Nov 2021 to Oct 2022 derived from the EGI accounting\cite{EGI2022} which uses the number of cores allocated to a pilot.   If four 4000 MB reconstruction jobs were sent to an 8-core pilot on a system with 2000MB/core, this would be equivalent to using 8 MWC (memory-weighted wall-time units).   


%\begin{table}[ht]
%\centering\csvautotabularright{external/CPUresources-2021-2022-2023-v3.csv}
%\caption{Summary  of DUNE wall-time pledges and contributions for 2021 and 2022.  The 2022 actual numbers are memory-weighted core-years.  Individual nations are listed and then merged (with US OSG) into a Global section.} 
%\label{tab:CPUUsage}
%\end{table}

%\begin{table}[ht]
%\centering\csvautotabularright{external/EIG-2022.csv}
%\centering\csvautotabularright{external/EIG-2022-Global.csv}
%\caption{Summary  of DUNE slot-years used for European collaborators, Nov. 21 to Oct. 22, using the EGI accounting\cite{EGI2022}.  The top shows all sites with EGI records, the bottom shows the summed contributions. These numbers reflect multiple-core slot reservations to account for larger memory use and are similar to, but  generally higher than, the FNAL accounting numbers in the previous table.} \label{tab:EIGSummary}
%\end{table}
%
%\begin{table}[ht]
%\centering
%\begin{tabular}{|l|l|l|}
%\hline
%Expt.&Max Memory& EGI VO Card\\
%\hline
%DUNE & 4000& \hrefII https://operations-portal.egi.eu/vo/view/voname/DUNE \\
%ATLAS & 2000& \hrefII https://operations-portal.egi.eu/vo/view/voname/atlas \\
%CMS & 2000& \hrefII https://operations-portal.egi.eu/vo/view/voname/cms \\
%LHCb & 4000& \hrefII https://operations-portal.egi.eu/vo/view/voname/lhcb \\
%ALICE& 2000& \hrefII https://operations-portal.egi.eu/vo/view/voname/alice \\
%BelleII&2048& \hrefII https://operations-portal.egi.eu/vo/view/voname/belle \\
%\hline
%\end{tabular}
%\caption{Maximum memory statements from the VO cards of major experiments.}
%\label{tab:VOcard}
%\end{table}
%
\pagebreak
\subsection{Conclusion}\label{sec:cpuresult} The advent of ProtoDUNE-2 running in 2024 and ramp-up of simulation for the FD and ND will lead to somewhat increased needs for CPU resources.  For 2024, we are requesting XXX kHS23-Yrs
%\section{
%\input{NearTerm_2024-02-02-2030_noMWC.tex}

%\section{Model Assumptions}

\cleardoublepage
%\renewcommand{\bibname}{References}
\bibliographystyle{utphys} 
\bibliography{bib}
\clearpage
\appendix

%\section{Information about storage from SAM}\label{storage}
%
%This section provides information on the sizes of data samples known to the SAM data catalog as of Nov. 1, 2022.  If a file has multiple copies, that is not shown here.  Tables \ref{tab:MCinSAM} and \ref{tab:DataInSam} show the total across all streams and data tiers while table \ref{tab:LargestSizes} shows the distribution of the largest samples.  
%
%
%\begin{table}[ht]
% \centering\csvautotabularright{external/mc.csv}
%\caption{Summary  of total simulation in SAM by detector type as of Nov 1, 2022.} 
%\label{tab:MCinSAM}
%\end{table}
%
%\begin{table}[ht]
% \centering\csvautotabularright{external/detector.csv}
% \caption{Summary  of total detector data in SAM by detector type as of Nov 1, 2022.}
% \label{tab:DataInSam}
%\end{table}
%
%
%
%\begin{table}[ht]
% \centering\csvautotabularright{external/TOPTYPES.csv}
%\caption{Classification of the largest data samples in SAM.  They are classified as detector(data) or mc, by the detector producing the data, by the stream (readout time) and by the data tier.  Some types, test and noise for example are archival only.  }
% \label{tab:LargestSizes}
%\end{table}
%\clearpage
\section{Model Details}

This appendix shows the parameters used in the model and plots of all the input and derived quantities as a function of time. 

Resource needs for reconstructed data for a given year are based on the number of events produced over the previous "Reprocess" years.   For ProtoDUNEs that is 2-4 years. 

Simulation resource needs are instead calculated based on a number of simulation events each year. The assumption is that new software versions imply resimulation.

Disk and tape lifetimes for different data types are specified as well as the desirable number of copies. 

The splits parameters make CERN responsible for raw data until 2027 with the collaboration taking over after that point. 

{\tt Detectors:} Detectors included in the calculation = {\tt ['SP', 'PDHD', 'DP', 'PDVD', 'HD', 'VD', 'ND-SAND', 'ND-LAr+TMS']} \\
{\tt Cap:} Cap on Raw data/year in PB = {\tt 30} \\
{\tt Base-Memory:} MB of memory per slot assumed as the average = {\tt 2000} \\
{\tt MaxYear:} Plot until year = {\tt 2030} \\
{\tt MinYear:} Plot starting with year = {\tt 2022} \\
{\tt Reprocess:} Number of years of data reprocessed when doing a new pass = {\tt {'SP': 3, 'DP': 3, 'PDHD': 3, 'PDVD': 3, 'PDs': 3, 'VD': 100, 'HD': 100, 'FDs': 100, 'ND-SAND': 100, 'ND-LAr+TMS': 100}} \\
{\tt AnalysisExtend:} Years analysis continues after last reco/sim = {\tt 3} \\
{\tt PatternFraction:} Fraction of time taken in pattern recognition = {\tt {'SP': 0.7, 'PDHD': 0.7, 'DP': 0.7, 'PDVD': 0.7, 'PDs': 0.7, 'HD': 0.1, 'VD': 0.1, 'FDs': 0.1, 'ND-SAND': 0.9, 'ND-LAr+TMS': 0.9, 'MARS': 0}} \\
{\tt TapeLifetimes:} Number of years kept on tape = {\tt {'Raw-Store': 100, 'Test': 1.0, 'Reco-Data-Store': 15, 'Sim-Store': 15}} \\
{\tt DiskLifetimes:} Number of years kept on disk = {\tt {'Raw-Store': 1, 'Test': 1.0, 'Reco-Data-Store': 3, 'Sim-Store': 2}} \\
{\tt TapeCopies:} Number of copies kept on tape = {\tt {'Raw-Store': 2, 'Test': 1, 'Reco-Data-Store': 1, 'Sim-Store': 1}} \\
{\tt DiskCopies:} Number of copies kept on disk = {\tt {'Raw-Store': 1, 'Test': 1, 'Reco-Data-Store': 2, 'Sim-Store': 1.5}} \\
{\tt PerYear:} Number of reprocessing done per year = {\tt {'Raw-Store': 1, 'Test': 1, 'Reco-Data-Store': 1, 'Sim-Store': 1, 'Events': 1, 'Sim-Events': 1, 'Reco-Data-CPU': 1, 'Sim-CPU': 1, 'Analysis': 1, 'Analysis-CPU': 1, 'Reco-Data-GPU': 1, 'Sim-GPU': 1}} \\
{\tt Cores:} Description of cores, efficiency and speed relative to 2020 vintage = {\tt {'Efficiency': 0.7, '2020Units': 1}} \\
{\tt kHEPSPEC06PerCPU:} kHEPSPEC06 per core assumed = {\tt 0.011} \\
{\tt SplitsYear:} Year CERN no longer responsible for disk or tape = {\tt 2029} \\
{\tt SplitsEarly:} Division between FNAL/CERN/Global for storage until SplitsYear = {\tt {'Tape': {'Raw-Store': {'FNAL': 0.5, 'CERN': 0.5, 'Global': 0.0}, 'Sim-Store': {'FNAL': 1.0, 'CERN': 0.0, 'Global': 0.0}, 'Reco-Data-Store': {'FNAL': 1.0, 'CERN': 0.0, 'Global': 0.0}, 'Test': {'FNAL': 0.5, 'CERN': 0.5, 'Global': 0.0}}, 'Disk': {'Raw-Store': {'FNAL': 0.5, 'CERN': 0.5, 'Global': 0.0}, 'Sim-Store': {'FNAL': 0.4, 'CERN': 0.1, 'Global': 0.5}, 'Reco-Data-Store': {'FNAL': 0.4, 'CERN': 0.1, 'Global': 0.5}, 'Test': {'FNAL': 0.5, 'CERN': 0.5, 'Global': 0.0}}, 'CPU': {'CPU': {'FNAL': 0.4, 'CERN': 0.1, 'Global': 0.5}}}} \\
{\tt SplitsLater:} Division between FNAL/CERN/Global for storage after SplitsYear = {\tt {'Tape': {'Raw-Store': {'FNAL': 0.5, 'CERN': 0.25, 'Global': 0.25}, 'Sim-Store': {'FNAL': 0.5, 'CERN': 0.0, 'Global': 0.5}, 'Reco-Data-Store': {'FNAL': 0.5, 'CERN': 0.0, 'Global': 0.5}, 'Test': {'FNAL': 0.5, 'CERN': 0.0, 'Global': 0.5}}, 'Disk': {'Raw-Store': {'FNAL': 1.0, 'CERN': 0.0, 'Global': 0.0}, 'Sim-Store': {'FNAL': 0.25, 'CERN': 0.0, 'Global': 0.75}, 'Reco-Data-Store': {'FNAL': 0.25, 'CERN': 0.0, 'Global': 0.75}, 'Test': {'FNAL': 0.5, 'CERN': 0.0, 'Global': 0.5}}, 'CPU': {'CPU': {'FNAL': 0.5, 'CERN': 0.0, 'Global': 0.5}}}} \\
{\tt filename:} Input configuration file = {\tt NearTerm\_2024-02-05-2040.json} \\


\end{document}
%\end{document}

